% In the beginning of the code, we declare the document type, as well as the packages to import and use
\documentclass{article}
\usepackage[utf8]{inputenc}
\usepackage[T1]{fontenc}
% Covington gives us the ability to have enumerated examples like in linguistics papers
\usepackage{covington}
% AVM allows us to code Attribute-Value Matrices like are used in HPSG
\usepackage{avm}
% Forest allows us to code syntax trees
\usepackage{forest}
\useforestlibrary{linguistics}
\forestapplylibrarydefaults{linguistics}
% You can change the page margins with the geometry package
\usepackage[margin=1in]{geometry}


% LaTeX will automatically make a document title based on the title, author, and date entered in these commands
\title{LaTeX Tutorial}
\author{Liz Conrad (ecconrad)}
\date{October 2021}


% This command begins the body of the LaTeX document. The document must end with \end{document}
\begin{document}

% As it suggests, this command makes the title block based on the title, author, and date you supplied
\maketitle


\section*{Chapter 1, Problem 1}

% Wrapping my answers in the "description" environment might make your document look a little bit neater, but this is really a preference thing
\begin{description}
    
    \item[(A)] The phrase for \textit{I am not a bear} in several languages can be seen in (\ref{bear-en})-(\ref{bear-ga}) below
    
    % When using the Covington package, make sure to wrap all your examples in the \begin{examples}...\end{examples} commands, so that they're numbered coherently. Each numbered Covington example must be started with \item
    \begin{examples}
        \item{I am not a bear}\label{bear-en}
        \item{nanuuŋŋittuŋa}\label{bear-iu}
        \item{én nem medve vagyok}\label{bear-hu}
        \item{ní béar mé}\label{bear-ga}
    \end{examples}
    
    \item[(B)] The tree for example 27 on page 34 (\textit{the cats like Sandy}) is shown in (\ref{tree})
    
    % Here is how to code a tree with the forest package
    \begin{examples}
    \item\begin{forest}
        [S 
            [NP 
                [D [the]]
                [NOM [N [cats]]]
            ]
            [VP
                [V [like]]
                [NP [NOM [N [Sandy]]]]
            ]
        ]
    \end{forest}\label{tree}
    \end{examples}
    
    \item[(C)] Here we see how AVMs look in LaTeX. A minimal AVM is shown in (\ref{minimal-avm}). The lexical entry for the pronoun \textit{you}, as found at the end of Chapter 3, is also shown in (\ref{you})
    
    % Here is how to code an AVM with the avm package
    % IMPORTANT: avm.sty must be in your project folder
    % Feel free to copy avm.sty from this project
    \begin{examples}
        %%begin novalidate%%
        \item\begin{avm}
            \[
                ATTRIBUTE 1 & value 1\\
                ATTRIBUTE 2 & value 2
            \]
        \end{avm}\label{minimal-avm}
        %%end novalidate
        
        %%begin novalidate
        \item\begin{avm}
        \< you, 
            \[
            \textit{word}\\
            HEAD & 
                \[
                \textit{noun}\\
                AGR & \[PER & 2nd\]
                \]\\
            VAL & 
                \avmbox{1}
                \[
                COMPS & itr\\
                SPR & +
                \]
            \]
        \>
        \end{avm}\label{you}
        %%end novalidate
    \end{examples}

\end{description}

% this command starts the following material on a new page
\newpage

\textbf{\Large Chapter 1, Problem 2}
\begin{description}
This is an example of a question that you might answer completely in prose, without the need for diagrams or subparts. You don't actually need the "description" flags around this part, but I find it looks a little better this way. LaTeX will also say that this looks to be missing an item, but it will still compile. \textbf{bold}, \textit{italic}, \textsc{small-caps}
\end{description}

\end{document}


\end{document}
